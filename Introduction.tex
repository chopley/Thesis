\chapter{Introduction}



\section{Science Goals}

\section{}

\section{Foreground Removal Techniques}

The beginnings CMB polarization studies focused on measuring the relatively bright E-Mode power spectrum. Small area surveys conveniently provided sufficient signal to detect both EE and TE polarization \cite{Leitch2005a,Mcmahon2005,Barkats2005,Readhead2004} , and areas obfuscated by foregrounds were avoided.  Detecting the much fainter B-Mode signals, will require larger area surveys and galactic foreground contamination will be unavoidable \cite{Tegmark2000}. This will require foreground removal.

A number of different foreground removal techniques exist. 


\cite{Hansen2006},\cite{Delabrouille2009},\cite{Leach2008},\cite{ArmitageCaplan2011},\cite{Gold2009},\cite{Dunkley2009},
\section{Survey Requirements}
  \subsection{Beam Size}
  \begin{eqnarray}
   \theta_{HPBW}&\approx& \frac{1.2\lambda}{D}\\
  \theta_{HPBW}&\approx& 0.68^{\circ}\\
   \theta_{HPBW}&\approx& 41''
  \end{eqnarray}

  \subsection{Confusion Limit}
At centimetre wavelengths the confusion noise ($\sigma_{c}$) is observed to be given by
\begin{eqnarray}
 \sigma_{c}&\approx& 0.2 (\nu)^{-0.7}(\theta_{HPBW})^2\\
 \sigma_{c}&\approx& 533mJy/beam
\end{eqnarray}


where the $\sigma_{c}$ is the confusion noise (mJ/Beam), $\nu$ is in GHz and $\theta_{HPBW}$ in arcminutes\\
\url{http://www.atnf.csiro.au/research/radio-school/2011/talks/CondonContinuum2011.pdf}\\
\url{http://www.cv.nrao.edu/course/astr534/Radiometers.html}\\


The confusion limit is about 5$\sigma_{c}$ or $\approx$ 2.5~Jy

